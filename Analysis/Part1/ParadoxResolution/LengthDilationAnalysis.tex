\section{Length dilation analysis}\label{sec:lengthDilation}
	It seems quite unfair to Alice to say that she can only calculate how much time passes for her by looking at Donald's clock and reversing the effect through some calculations.
	After all, Alice does have a clock on her ship.
	Why can't she use that clock?

	She can, but I've saved this approach it until now because it involves another way of looking at the problem and introduces a new concept: length dilation.

	Just as time dilates for objects moving quickly, so does length, and this is represented by the following formula \autocite[\ppno~588--594]{textbook}:
	\begin{equation}\label{eq:lengthDilation}
		L_m = L_s \sqrt{1 - \frac{v^2}{\si{\clight}^2}}
	\end{equation}
	Here, $L_m$ represents the length of an object as measured by an observer moving relative to it at a speed $v$.
	The object has a length of $L_s$ when measured by someone at rest relative to the object.

	\subsection{How does this apply to Alice?}
		Alice is in two different inertial frames of reference during her trip.
		She is in one inertial frame of reference during the outbound leg of her trip, and she is in another one on the inbound leg of her trip.
		Recall that special relativity states that the laws of physics are the same in all inertial frames of reference \autocite{textbook,einstein1916}.
		
		So, we can say that Alice is at rest, and the entire universe is moving relative to her.
		This allows us to apply length dilation to the universe.

		We will call $L_U$ the length of the portion of the universe Alice is travelling through.
		The universe measures this length, $L_U$, as \SI{4}{\lightyear}.
		However, since the universe is moving relative to Alice, she will measure a different length, which we will call $L_A$.

		We replace $L_m$ from \eqref{eq:lengthDilation} with $L_A$, $L_s$ with $L_U$, and $\frac{v^2}{\si{\clight}}$ with $\beta$, then calculate $L_A$:
		\begin{align*}
			L_A &= L_U \sqrt{1 - \beta^2}\\
			L_A &= \SI{4}{\lightyear} \sqrt{1 - 0.8^2}\\
			L_A &= \SI{4}{\lightyear} \sqrt{\frac{9}{25}}\\
			L_A &= \SI{4}{\lightyear} \times \frac{3}{5}\\
			L_A &= \SI{2.4}{\lightyear}
		\end{align*}
		Alice measures the length of each leg of her trip as \SI{2.4}{\lightyear}.
		She knows her speed is \SI{0.8}{\clight}, and so she can easily calculate how long one leg of her journey will take:
		\begin{align*}
			\Del t_A &= \frac{d}{v}\\
			\Del t_A &= \frac{\SI{2.4}{\lightyear}}{\SI{0.8}{\clight}}\\
			\Del t_A &= \SI{3}{\year}
		\end{align*}
		So, Alice calculates that each leg of her journey takes three years.

		Again, there are two legs of her journey, $3 + 3 = 6$, so her entire journey will take 6 years.
	\subsection{The problem with the length dilation analysis}
		While this analysis was a lot simpler than our previous analysis with Minkowski spacetime and the relativistic Doppler effect, it is not as robust.
		This analysis only works in an inertial frame of reference.
		In practice, Alice will undergo acceleration, and the length dilation analysis no longer holds for that.

		Next, we will show that our original analysis also works if Alice undergoes acceleration.
