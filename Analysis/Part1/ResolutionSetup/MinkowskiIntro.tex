\section{Minkowski spacetime diagrams and world lines}
	We must now introduce a new way of analyzing problems in special relativity: the Minkowski spacetime diagrams and world lines.

	We will draw the world lines of our situation onto Minkowski spacetime diagrams.
	This is simply a diagram with time on the vertical axis, and distance on the horizontal axis.

	We will start with a spacetime diagram of what we would expect to see, disregarding special relativity, using Donald as the frame of reference. See figure \vref{fig:classicalMinkowski}.
	\begin{figure}[H]
		\begin{minipage}{0.3\textwidth}
			\caption{Minkowski spacetime diagram of what we would expect to see according to classical physics with Donald as the frame of reference.}
			\label{fig:classicalMinkowski}
		\end{minipage}
		\hfill
		\begin{minipage}{0.6\textwidth}
			\begin{tikzpicture}
				\begin{axis}[
						xlabel=$d\ (\si{\lightyear})$,
						ylabel=$t\ (\si{\year})$,
						xmin=0,
						xmax=4.25,
						ymin=0,
						ymax=10.75,
						xtick distance=1,
						ytick distance=1,
						height=4in,
						width=4in,
					]
					\addplot plot coordinates {
					(0,0)
					(4,5)
					(0,10)
					};
				\end{axis}
			\end{tikzpicture}
		\end{minipage}
	\end{figure}
	Of course, figure \vref{fig:classicalMinkowski} does not apply, since $\beta = 0.8$ (which is quite large), however, I find it helpful to visualize the problem and introduce the notion of spacetime diagrams.
