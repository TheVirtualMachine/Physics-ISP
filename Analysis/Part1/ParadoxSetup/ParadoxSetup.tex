\chapter{Stating the Twin Paradox}
	\section{The problem}
		Imagine two people, we will call them Alice\footnote{Of computer security white paper fame.} and Donald\footnote{Of presidential infamy.}.

		Donald is currently the president of the United States, and he is doing a bad job.
		Alice doesn't like this situation, so she is going to to time travel to the future when Donald is no longer president.
	\section{Solution to Trump}
		Special relativity tells us that time will pass slower for an object in motion than an object at rest \autocites{textbook}{einstein1916}.
		This means that if Alice moves very fast, she will experience less time pass for her than on earth, and will effectively time travel to the future.
	\section{The paradox}
		The paradox here is that special relativity states that the laws of physics are the same in all inertial frames of reference \autocites{textbook}{einstein1916}.
		This means that we could also argue that Alice is not moving, and instead the earth is moving, which would result in Alice aging faster than Donald, which is the opposite of what we want and the opposite of what actually happens \autocite[\ppno~593--594]{textbook}.

		This is called the twin paradox.
	\section{Approach to resolution of the paradox}
		It is commonly thought that general relativity is needed to resolve this paradox because Alice is in an accelerating frame of reference and special relativity cannot handle accelerating frames of reference.
		That is not true.
		Special relativity can indeed handle accelerating frames of reference, but it is more difficult \autocites{sracceleration}{twinparadox}.
		However, it is much easier to use special relativity to solve this problem than it is to use general relativity.
		And in fact, we will later see that the acceleration of Alice is irrelevant to the resolution of the paradox.
	\section{Notation and assumptions}
		\subsection{Notation}\label{subsec:notation}
			\subsubsection{Our notation}
				In math, Alice will be referred to as $A$, and Donald will be referred to as $D$.
			\subsubsection{Other physics notation}
				Following is some common non-trivial physics notation that we will be using:
				\paragraph{Velocity}
					\[v = \|\vec{v}\|\]
					We will use $v$ to denote $\|\vec{v}\|$.
				\paragraph{Acceleration}
					\[a = \|\vec{a}\|\]
					We will use $a$ to denote $\|\vec{a}\|$.
				\paragraph{Speed of light}
					\[\si{\clight} = \SI{299792458}{\metre/\second}\]
					We will use \si{\clight} as the speed of light in a vacuum. We are using \si{\clight} instead of $c$ because \si{\clight} is the recommended SI notation \autocite{siunits}.
				\paragraph{$\beta$ notation}
					\[\beta = \frac{\varSI{v}{\metre/\second}}{\varSI{\clight}{\metre/\second}}\]
					Where $v$ is velocity and $\si{\clight}$ is the speed of light. Also, since nothing can exceed or meet the speed of light, and the direction is not relevant to the amount of time dilation \autocite{textbook,einstein1916}, we will say that: $0 \leq \beta < 1$.
		\subsection{Assumptions}\label{subsec:twinAssumptions}
			\subsubsection{Values}
				We will say that Alice travels a distance of 4 light years at a speed of \SI{0.8}{\clight} since these are nice numbers to work with \autocite[\pno~35]{kogut2012introduction}.
				She then turns around and comes back to earth.
			\subsubsection{Turnaround}
				We will assume that Alice makes an instantaneous turnaround.
				Later we will show that this assumption has no effect on the resolution of the paradox.
		\subsection{Other terminology}
			Alice's trip is split up into two parts.
			First, she is moving away from earth, which we will call the outbound leg of the trip.
			After, she is moving towards the earth, which we will call the inbound leg of the trip.
