\section{Deriving the formulas}
	Before making any calculations with velocity, we must first redefine our equation for velocity. Given that the speed of
	light, \si{\clight}, is constant and is considered the ``universal speed limit'', it can be reasoned that $v=at$ will no longer serve
	as an accurate calculation for velocity since it can exceed \si{\clight}. Therefore we must use $v=\si{\clight} \tanh\left(\frac{at}{\si{\clight}}\right)$ as our equation for 
	velocity, as it is the ``true acceleration formula'' \autocite{sracceleration}.

	For these calculations to be accurate, we must also assume that the earth is stationary, and that there are no other existing significant
	gravitational masses other than the earth. Additionally, we must ignore gravitational time dilation.

	It should also be noted that the dilation values calculated will be approximations calculated by a written computer simulation. This
	simulation will sum the time dilation across each 1 second interval for the entirety of the journey, which will not yield a perfect
	result, but the calculated result will be accurate enough for our purposes.
