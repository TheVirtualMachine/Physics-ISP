\section{The relativistic Doppler effect}
	The simplest way to solve the twin paradox is to use the relativistic Doppler effect in order to analyze what Donald sees and what Alice sees.

	We will have both Alice and Donald flash a light at the other once per second, according to their own proper time.
	Their lights are infinitely powerful, and can be seen from light years away (once the light has travelled there of course).
	We will assume the path of the lights are entirely through a perfect vacuum.
	\subsection{More notation}
		First, let's define some notation specific to this section:
		\begin{description}
			\item[$f_s$] The frequency the emitter (source) flashes their light at.
			\item[$f_o$] The frequency the other person (observer) sees the light flashing at.
			\item[$\lam$] The distance to the next wave front of the approaching light wave. $\lam$ is calculated as:
			              \[\lam = \frac{\varSI{\clight}{\metre/\second}}{\varSI{f_s}{\per\second}} = \varSI{\frac{\si{\clight}}{f_s}}{\metre}\]
		\end{description}
	\subsection{Relativistic Doppler effect derivation}
		For now, we will make Alice the source of the flashes, and Donald the observer.
		We could do it the other way around, but we will soon see that it does not have any effect on the fully derived formula, as both $\Del t_A$ and $\Del t_D$ will cancel out.

		We start by relating $\Del t_D$ to $\lam$ and $v$ when Alice is moving away from Donald:
		\begin{align*}
			\Del t_D &= \frac{\lam}{\si{\clight}} + \frac{v \times \Del t_D}{\si{\clight}}\\
			\si{\clight} \Del t_D &= \lam + v\Del t_D\\
			\si{\clight} \Del t_D - v\Delta t_D &= \lam\\
			\Del t_D (\si{\clight} - v) &= \lam\\
			\Del t_D &= \frac{\lam}{\si{\clight} - v}\\
			\Del t_D &= \frac{1}{\si{\clight} - v} \times \lam
		\end{align*}
		Now substitute in $\lam = \frac{\si{\clight}}{f_s}$:
		\begin{align}
			\Del t_D &= \frac{1}{\si{\clight} - v} \times \frac{\si{\clight}}{f_s}\nonumber\\
			\Del t_D &= \frac{\si{\clight}}{\si{\clight} - v} \times \frac{1}{f_s}\nonumber\\
			\Del t_D &= \frac{1}{1 - \frac{v}{\si{\clight}}} \times \frac{1}{f_s}\nonumber\\
			\Del t_D &= \frac{1}{f_s(1 - \beta)}\label{eq:dop1}
		\end{align}
		Next, we will perform a unit analysis to verify that \eqref{eq:dop1} gives us a value in seconds:
		\begin{align*}
			\Del t_D &= \frac{1}{\si{\per\second}}\\
			\Del t_D &= \si{\second}
		\end{align*}
		So, we have derived \eqref{eq:dop1} and verified that it gives us a value in seconds.
		Now we need to use this formula.

		The next step is to develop the actual Doppler effect formula.
		We will work off of the special relativity time dilation formula \autocite[\pno~593]{textbook}:
		\begin{equation}\label{eq:dopTimeDilation1}
			\Del t_D = \frac{\Del t_A}{\sqrt{1 - \beta^2}}
		\end{equation}
		We will now substitute \eqref{eq:dop1} into \eqref{eq:dopTimeDilation1} to combine our two equations in order to develop a third formula:
		\begin{align}
			\Del t_D &= \frac{\Del t_A}{\sqrt{1 - \beta^2}}\nonumber\\
			\frac{1}{f_s(1 - \beta)} &= \frac{\Del t_A}{\sqrt{1 - \beta^2}}\nonumber\\
			\Del t_A &= \frac{\sqrt{1 - \beta^2}}{f_s(1 - \beta)}\label{eq:dopTimeDilation2}
		\end{align}
		Next, we will finish developing the relativistic Doppler shift formula.

		Note that, by definition:
		\begin{equation}\label{eq:dop2}
			f_o = \frac{1}{\Del t_D}
		\end{equation}
