\section{Why is speed required for earth travel and acceleration for space travel?}
	When it comes to running, an athlete reaches their top speed within a few seconds. Thus, the time spent accelerating is minimal and does not affect the calculation for time skipped by a significant margin. Although aircraft spend slightly more time accelerating, the same can be said because the value is small enough that the calculations are not affected. Since acceleration can be ignored, the simple textbook formula \autocite{textbook} for time dilation can be used.

	With space travel, things are slightly different. In order to reach speeds at which time dilation will have a noticeable effect, the ships must spend a significant amount of time accelerating. Unlike the previous cases, this will impact the calculations, and different mathematics must be used. The computer simulation is used to calculate these cases.

	Keep in mind that when we refer to time skipped, we are actually referring to the difference in elapsed time that the two observers experience.
