\section{Space shuttling}
	Travelling through space while accelerating immensely complicates matters, and transforms our nice equation into a humongous monstrosity of constants and hyperbolic trigonometric functions. In order to solve for the amount of time skipped, we will be using the computer simulation. The only values needed to feed into the simulation are the time remaining in Donald's presidency and the acceleration of the space shuttle. Recalling that $\Del t_D = \SI{99779400}{\second}$ and that $a_{space\ shuttle} = \SI{15.25}{\metre/\second^2}$, we run the simulation. The results are as follows:

	For the curved path of the space shuttle, Alice is travelling in her frame of reference for \num{84694161} seconds, or about two years, eight months, and seven days. The amount of time Alice misses of Donald’s presidency (time skipped) is about \SI{1.508524131e7}{\second}, or about five months and 22 days. Alice will arrive approximately two seconds after Donald’s last moments in office.

As for the linear path of the space shuttle, Alice is travelling in her frame of reference for \num{83301044} seconds, or about two years, seven months, and 21 days. The amount of time Alice skips of Donald’s presidency is around \SI{1.647835804e7}{\second}, or about six months and eight days. Alice will again arrive about two seconds after Donald’s last moments in office.

As shown, the linear path is more optimal than the curved path, as more time is skipped and less time is spent travelling, relative to Alice’s frame of reference. In addition, the amount of time skipped is actually significant, unlike when Alice was running and flying. If Alice were able to travel in the space shuttle as described, then this mode of transportation would be meaningful.
