\section{Rocketing}
	The next form of transport breaks away from the barriers of earth and into the realm of space travel. Thus, we will be needing to find acceleration rather than maximum speed. In \cite {rocketAcceleration}, the acceleration of the Space Shuttle \textit{Discovery} is calculated by dividing the net force by the total mass of the rocket, which is two million kilograms. \cite{rocketAcceleration} calculates the net force by subtracting the thrust force of 30.5 meganewtons by the gravitational force on the space shuttle. Since the space shuttle will be flying through space, which we will assume is empty, the gravitational force can be ignored. This means the net force is \SI{30.5}{\MN}. We can use a rearranged version Newton's second law, $a = \frac{F}{m}$, to solve for acceleration. This gives us a value of $\frac{\SI{3.05e7}{\newton}}{\SI{2.00e6}{\kg}}$, or \SI{15.25}{\metre/\second^2}.
