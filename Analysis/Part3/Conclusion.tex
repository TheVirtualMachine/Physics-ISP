\chapter{Conclusion about the forms of travel}
	Out of the four modes of transportation discussed: running, flying, space shuttling, and using a black hole starship; some stand out from the others in terms of practicality. Although running and flying are the easiest to do, the amount of time skipped is so small that these forms of travel can be completely ignored with regards to our goal of skipping a meaningful amount of time. Using a space shuttle and black hole starship, on the other hand, can actually get us significant numbers.

	Although the space shuttle and black hole starship have similar acceleration values, as well as similar values for total time skipped, the black hole starship is slightly more effective. In our calculations, we assumed that the thrust force of the space shuttle, and the acceleration as a result, could be maintained indefinitely. In reality, the space shuttle would run out of fuel not far into its journey, as space shuttles are not designed for long distance travelling. If, somehow, enough fuel was to be stored on the ship to last the duration of the trip, the heavily increased mass of the ship due to having more fuel would affect the acceleration and make the trip impossible yet again.

	As a result, the black hole starship is actually more feasible than the space shuttle for our purposes because a black hole is almost an unlimited source of fuel. The issue with the starship however, is that it is theoretical and may not even work. Even the first step of creating a long-lasting microscopic black hole is well out of reach of our current technology.

	In conclusion, with our current technology, there is no way to skip a meaningful amount of time of Donald's presidency using special relativity. We can either deal with Donald being president for a few more years, or hope that technology can create a long-lasting microscopic black hole soon.
