\section{Deriving the relativistic Doppler effect}
	The simplest way to solve the twin paradox is to use the relativistic Doppler effect in order to analyze what Donald sees and what Alice sees.

	We will have both Alice and Donald flash a light at the other once per second, according to their own proper time.
	Their lights are infinitely powerful, and can be seen from light years away (once the light has travelled there of course).
	We will assume the path of the lights are entirely through a perfect vacuum.
	\subsection{More notation}
		First, let's define some notation specific to this section:
		\begin{description}
			\item[$f_s$] The frequency the emitter (source) flashes their light at. We will make Alice the source.
			\item[$f_o$] The frequency the other person (observer) sees the light flashing at. We will make Donald the observer of Alice's flashing light.
			\item[$t_s$] The proper time to the next wavefront as seen by the emitting source, Alice.
			\item[$t_o$] The proper time to the next wavefront as seen by person observing the flashes, Donald.
			\item[$\lam$] The distance to the next wave front of the approaching light wave. $\lam$ is calculated as:
			              \[\lam = \frac{\varSI{\clight}{\metre/\second}}{\varSI{f_s}{\per\second}} = \varSI{\frac{\si{\clight}}{f_s}}{\metre}\]
		\end{description}
	\subsection{The derivation}
		We start by relating $\Del t_s$ to $\lam$ and $v$ when Alice is moving away from Donald:
		\begin{align*}
			\Del t_s &= \frac{\lam}{\si{\clight}} + \frac{v \times \Del t_s}{\si{\clight}}\\
			\si{\clight} \Del t_s &= \lam + v\Del t_s\\
			\si{\clight} \Del t_s - v\Delta t_s &= \lam\\
			\Del t_s (\si{\clight} - v) &= \lam\\
			\Del t_s &= \frac{\lam}{\si{\clight} - v}\\
			\Del t_s &= \frac{1}{\si{\clight} - v} \times \lam
		\end{align*}
		Now substitute in $\lam = \frac{\si{\clight}}{f_s}$:
		\begin{align}
			\Del t_s &= \frac{1}{\si{\clight} - v} \times \frac{\si{\clight}}{f_s}\nonumber\\
			\Del t_s &= \frac{\si{\clight}}{\si{\clight} - v} \times \frac{1}{f_s}\nonumber\\
			\Del t_s &= \frac{1}{1 - \frac{v}{\si{\clight}}} \times \frac{1}{f_s}\nonumber\\
			\Del t_s &= \frac{1}{f_s(1 - \beta)}\label{eq:dop1}
		\end{align}
		Next, we will perform a unit analysis to verify that \eqref{eq:dop1} gives us a value in seconds:
		\begin{align*}
			\Del t_s &= \frac{1}{\si{\per\second}}\\
			\Del t_s &= \si{\second}
		\end{align*}
		So, we have derived \eqref{eq:dop1} and verified that it gives us a value in seconds.
		Now we need to use this formula.

		The next step is to develop the actual Doppler effect formula.
		We will work off of the special relativity time dilation formula given in \cite[\pno~593]{textbook}, modified to use our notation, and rearrange it into a form more useful for our purposes:
		\begin{align}
			\Del t_s &= \frac{\Del t_o}{\sqrt{1 - \beta^2}}\nonumber\\
			\Del t_o &= \Del t_s\sqrt{1 - \beta^2}\label{eq:dopTimeDilation1}
		\end{align}
		We will now substitute \eqref{eq:dop1} into \eqref{eq:dopTimeDilation1} to combine our two equations in order to develop a third formula:
		\begin{equation}\label{eq:dop3}
			\Del t_o = \frac{\sqrt{1 - \beta^2}}{f_s(1 - \beta)}
		\end{equation}
		Next, we will finish developing the relativistic Doppler shift formula.

		Note that, by definition:
		\begin{equation}\label{eq:dop2}
			f_o = \frac{1}{\Del t_o}
		\end{equation}
		We will now substitute \eqref{eq:dop3} into \eqref{eq:dop2}:
		\begin{align}
			f_o &= \frac{1}{\frac{\sqrt{1 - \beta^2}}{f_s(1 - \beta)}}\nonumber\\
			f_o &= \frac{f_s(1 - \beta)}{\sqrt{1 - \beta^2}}\nonumber\\
			f_o &= f_s\left(\frac{1 - \beta}{\sqrt{1 - \beta^2}}\right)\nonumber\\
			f_o &= f_s\left(\frac{\sqrt{(1 - \beta)^2}}{\sqrt{1^2 - \beta^2}}\right)\nonumber\\
			f_o &= f_s\left(\frac{\sqrt{(1 - \beta)(1 - \beta)}}{\sqrt{(1 - \beta)(1 + \beta)}}\right)\nonumber\\
			f_o &= f_s\sqrt{\frac{(1 - \beta)(1 - \beta)}{(1 - \beta)(1 + \beta)}}\nonumber\\
			f_o &= f_s \times \sqrt{\frac{1 - \beta}{1 + \beta}}\label{eq:relDopAway}
		\end{align}
		This is the formula for the relativistic Doppler effect when Alice and Donald are moving away from each other.
		If Alice is moving towards Donald, then we simply change the sign on $\beta$ to get:
		\begin{equation}\label{eq:relDopTowards}
			f_o = f_s \times \sqrt{\frac{1 + \beta}{1 - \beta}}
		\end{equation}
