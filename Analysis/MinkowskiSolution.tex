\section{Minkowski spacetime, the relativistic Doppler effect, and the twin paradox}
	Recall the assumptions and values we decided to use in subsection \ref{subsec:twinAssumptions}:
	\begin{description}
		\item[Distance] Alice travels a distance of 4 light years, so $d = \SI{4}{\lightyear}$.
		\item[Velocity] Alice travels at a speed of \SI{0.8}{\clight}, so $v = \SI{0.8}{\clight}$ and $\beta = 0.8$.
	\end{description}
	The actual values we use do not matter for resolving the paradox.
	These values were chosen because they give nice numbers when we perform the calculations, which makes the analysis easier to follow.

	We also assumed an instantaneous turnaround.
	This too makes the math easier, but we will show that it does not change the resolution to the paradox.

	Analyzing the twin paradox with the relativistic Doppler effect is helpful because it allows us to calculate what each person sees, and show that they are seeing different things, which solves the ambiguity stated in section \ref{sec:naive}.
