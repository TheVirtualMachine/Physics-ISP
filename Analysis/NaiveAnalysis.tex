\section{A naive analysis}
	We have:
	\[v = \SI{0.8}{\clight} \implies \beta = 0.8\]
	\[d = \SI{4}{\lightyear}\]
	And we know this formula from \cite[\pno~583]{textbook} (modified to use our notation):
	\[\Delta t_D = \frac{\Delta t_A}{\sqrt{1 - \beta}}\]
	Which can be rearranged into:
	\begin{equation}\label{eq:aliceTime}
		\Delta t_A = \Delta t_D\sqrt{1-\beta^2}
	\end{equation}
	We can trivially calculate how much time should pass for Donald:
	\[\Delta t_D = \frac{2d}{v} = \frac{2 \times \SI{4}{\lightyear}}{\SI{0.8}{\clight}} = \frac{\SI{8}{\lightyear}}{\SI{0.8}{\clight}} = \SI{10}{\year}\]
	And how much time should pass for Alice follows by simply plugging this into \eqref{eq:aliceTime}:
	\begin{align*}
		\Delta t_A &= \Delta 10\sqrt{1-0.8^2}\\
		\Delta t_A &= \Delta 10\sqrt{\frac{9}{25}}\\
		\Delta t_A &= \Delta 10\frac{3}{5}\\
		\Delta t_A &= 6
	\end{align*}
	So, Donald ages by 10 years, and Alice ages by 6 years.

	This answer is right, but the problem is that we started by assuming that Donald is stationary and Alice is moving.
	However, we could have said that Alice is stationary and Donald is moving, and then we would calculate $\Delta t_A = 10$ and $\Delta t_D = 6$, which is wrong.
	So doing the analysis this way leads to ambiguity.
	We must develop a more rigorous way to analyze this problem.
