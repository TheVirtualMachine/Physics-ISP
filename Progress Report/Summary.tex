\chapter{Summary of Topic}
	\section{Special relativity}
		We will primarily be investigating time dilation in special relativity.

		Put \emph{very} simply, special relativity tells us that if we move very fast, we time travel into the future. (Of course, that is by no means a rigorous --- or entirely accurate --- explanation. We will explain it more fully in subsection \ref{summary:formula}.)

		One of the main equations we will use is:
		\[\frac{\Delta t_m}{\Delta t_s} = \frac{1}{\sqrt{1 - \frac{v^2}{c^2}}}\]
		Again, put simply, this equation relates the ratio of elapsed time between two observers to the speed of one of them. Following is a more rigorous explanation.
		\subsection{What is that formula?}
			\label{summary:formula}
			\[\frac{\Delta t_m}{\Delta t_s} = \frac{1}{\sqrt{1 - \frac{v^2}{c^2}}}\]
			We will not go into proving this formula here, as that is beyond the scope of this text. Look at \parencite[p.~580 -- 586]{textbook} for a simple explanation or \parencite{einstein1916} for a more rigorous one. For now, we will just explain what all of this means.

			Imagine two people, each holding a clock. We will call these people $m$ and $s$.

			$m$ is a person standing still relative to an inertial frame of reference (such as the earth\footnote{The earth is not technically an inertial frame of reference, but it is close enough to one for our purposes.}).

			$s$ is a person moving with a velocity of $v$~\si{\m/\s} relative to $m$.

			$c$ is the speed of light in \si{\m/\s}.

			Then, $\Delta t_m$ is how much time person $m$ records on their clock, and $\Delta t_s$ is how much time person $s$ records on their clock after $s$ travels a certain distance relative to $m$ (The actual distance is not important, since we are just looking at a ratio. However, if it helps to visualize it, you can imagine that person $s$ is travelling for 1 second, making the distance equal to the magnitude of $v$.).
	\section{Other topics}
		We will also touch on other topics, such as kinematics, to do our calculations.

		For example, we will need to calculate how long it takes to accelerate to a given percentage of $c$ assuming an acceleration no greater than what a human can handle for a sustained period of time.

		Depending on where our analysis goes, we may also look at: general relativity, centripetal motion, basic orbital mechanics, or basic rocket science.
